% Tabela gerada pelo R com pequenas modificações feitas manualmente
\begin{table}[H]
\centering
\scriptsize
\begin{tabular}{rcccccccccccccccccccccc}
  \hline
 & \multicolumn{22}{c}{$\nu_1$} \\

$\nu_2$ & 1 & 2 & 3 & 4 & 5 & 6 & 7 & 8 & 9 & 10 & 11 & 12 & 13 & 14 & 15 & 16 & 18 & 20 & 30 & 40 & 60 & 120 \\ 
  \hline
2 & 8.53 & 9.00 & 9.16 & 9.24 & 9.29 & 9.33 & 9.35 & 9.37 & 9.38 & 9.39 & 9.40 & 9.41 & 9.41 & 9.42 & 9.42 & 9.43 & 9.44 & 9.44 & 9.46 & 9.47 & 9.47 & 9.48 \\ 
  3 & 5.54 & 5.46 & 5.39 & 5.34 & 5.31 & 5.28 & 5.27 & 5.25 & 5.24 & 5.23 & 5.22 & 5.22 & 5.21 & 5.20 & 5.20 & 5.20 & 5.19 & 5.18 & 5.17 & 5.16 & 5.15 & 5.14 \\ 
  4 & 4.54 & 4.32 & 4.19 & 4.11 & 4.05 & 4.01 & 3.98 & 3.95 & 3.94 & 3.92 & 3.91 & 3.90 & 3.89 & 3.88 & 3.87 & 3.86 & 3.85 & 3.84 & 3.82 & 3.80 & 3.79 & 3.78 \\ 
  5 & 4.06 & 3.78 & 3.62 & 3.52 & 3.45 & 3.40 & 3.37 & 3.34 & 3.32 & 3.30 & 3.28 & 3.27 & 3.26 & 3.25 & 3.24 & 3.23 & 3.22 & 3.21 & 3.17 & 3.16 & 3.14 & 3.12 \\ 
  6 & 3.78 & 3.46 & 3.29 & 3.18 & 3.11 & 3.05 & 3.01 & 2.98 & 2.96 & 2.94 & 2.92 & 2.90 & 2.89 & 2.88 & 2.87 & 2.86 & 2.85 & 2.84 & 2.80 & 2.78 & 2.76 & 2.74 \\ 
  7 & 3.59 & 3.26 & 3.07 & 2.96 & 2.88 & 2.83 & 2.78 & 2.75 & 2.72 & 2.70 & 2.68 & 2.67 & 2.65 & 2.64 & 2.63 & 2.62 & 2.61 & 2.59 & 2.56 & 2.54 & 2.51 & 2.49 \\ 
  8 & 3.46 & 3.11 & 2.92 & 2.81 & 2.73 & 2.67 & 2.62 & 2.59 & 2.56 & 2.54 & 2.52 & 2.50 & 2.49 & 2.48 & 2.46 & 2.45 & 2.44 & 2.42 & 2.38 & 2.36 & 2.34 & 2.32 \\ 
  9 & 3.36 & 3.01 & 2.81 & 2.69 & 2.61 & 2.55 & 2.51 & 2.47 & 2.44 & 2.42 & 2.40 & 2.38 & 2.36 & 2.35 & 2.34 & 2.33 & 2.31 & 2.30 & 2.25 & 2.23 & 2.21 & 2.18 \\ 
  10 & 3.29 & 2.92 & 2.73 & 2.61 & 2.52 & 2.46 & 2.41 & 2.38 & 2.35 & 2.32 & 2.30 & 2.28 & 2.27 & 2.26 & 2.24 & 2.23 & 2.22 & 2.20 & 2.16 & 2.13 & 2.11 & 2.08 \\ 
  11 & 3.23 & 2.86 & 2.66 & 2.54 & 2.45 & 2.39 & 2.34 & 2.30 & 2.27 & 2.25 & 2.23 & 2.21 & 2.19 & 2.18 & 2.17 & 2.16 & 2.14 & 2.12 & 2.08 & 2.05 & 2.03 & 2.00 \\ 
  12 & 3.18 & 2.81 & 2.61 & 2.48 & 2.39 & 2.33 & 2.28 & 2.24 & 2.21 & 2.19 & 2.17 & 2.15 & 2.13 & 2.12 & 2.10 & 2.09 & 2.08 & 2.06 & 2.01 & 1.99 & 1.96 & 1.93 \\ 
  13 & 3.14 & 2.76 & 2.56 & 2.43 & 2.35 & 2.28 & 2.23 & 2.20 & 2.16 & 2.14 & 2.12 & 2.10 & 2.08 & 2.07 & 2.05 & 2.04 & 2.02 & 2.01 & 1.96 & 1.93 & 1.90 & 1.88 \\ 
  14 & 3.10 & 2.73 & 2.52 & 2.39 & 2.31 & 2.24 & 2.19 & 2.15 & 2.12 & 2.10 & 2.07 & 2.05 & 2.04 & 2.02 & 2.01 & 2.00 & 1.98 & 1.96 & 1.91 & 1.89 & 1.86 & 1.83 \\ 
  15 & 3.07 & 2.70 & 2.49 & 2.36 & 2.27 & 2.21 & 2.16 & 2.12 & 2.09 & 2.06 & 2.04 & 2.02 & 2.00 & 1.99 & 1.97 & 1.96 & 1.94 & 1.92 & 1.87 & 1.85 & 1.82 & 1.79 \\ 
  16 & 3.05 & 2.67 & 2.46 & 2.33 & 2.24 & 2.18 & 2.13 & 2.09 & 2.06 & 2.03 & 2.01 & 1.99 & 1.97 & 1.95 & 1.94 & 1.93 & 1.91 & 1.89 & 1.84 & 1.81 & 1.78 & 1.75 \\ 
  17 & 3.03 & 2.64 & 2.44 & 2.31 & 2.22 & 2.15 & 2.10 & 2.06 & 2.03 & 2.00 & 1.98 & 1.96 & 1.94 & 1.93 & 1.91 & 1.90 & 1.88 & 1.86 & 1.81 & 1.78 & 1.75 & 1.72 \\ 
  18 & 3.01 & 2.62 & 2.42 & 2.29 & 2.20 & 2.13 & 2.08 & 2.04 & 2.00 & 1.98 & 1.95 & 1.93 & 1.92 & 1.90 & 1.89 & 1.87 & 1.85 & 1.84 & 1.78 & 1.75 & 1.72 & 1.69 \\ 
  19 & 2.99 & 2.61 & 2.40 & 2.27 & 2.18 & 2.11 & 2.06 & 2.02 & 1.98 & 1.96 & 1.93 & 1.91 & 1.89 & 1.88 & 1.86 & 1.85 & 1.83 & 1.81 & 1.76 & 1.73 & 1.70 & 1.67 \\ 
  20 & 2.97 & 2.59 & 2.38 & 2.25 & 2.16 & 2.09 & 2.04 & 2.00 & 1.96 & 1.94 & 1.91 & 1.89 & 1.87 & 1.86 & 1.84 & 1.83 & 1.81 & 1.79 & 1.74 & 1.71 & 1.68 & 1.64 \\ 
  21 & 2.96 & 2.57 & 2.36 & 2.23 & 2.14 & 2.08 & 2.02 & 1.98 & 1.95 & 1.92 & 1.90 & 1.87 & 1.86 & 1.84 & 1.83 & 1.81 & 1.79 & 1.78 & 1.72 & 1.69 & 1.66 & 1.62 \\ 
  22 & 2.95 & 2.56 & 2.35 & 2.22 & 2.13 & 2.06 & 2.01 & 1.97 & 1.93 & 1.90 & 1.88 & 1.86 & 1.84 & 1.83 & 1.81 & 1.80 & 1.78 & 1.76 & 1.70 & 1.67 & 1.64 & 1.60 \\ 
  23 & 2.94 & 2.55 & 2.34 & 2.21 & 2.11 & 2.05 & 1.99 & 1.95 & 1.92 & 1.89 & 1.87 & 1.84 & 1.83 & 1.81 & 1.80 & 1.78 & 1.76 & 1.74 & 1.69 & 1.66 & 1.62 & 1.59 \\ 
  24 & 2.93 & 2.54 & 2.33 & 2.19 & 2.10 & 2.04 & 1.98 & 1.94 & 1.91 & 1.88 & 1.85 & 1.83 & 1.81 & 1.80 & 1.78 & 1.77 & 1.75 & 1.73 & 1.67 & 1.64 & 1.61 & 1.57 \\ 
  25 & 2.92 & 2.53 & 2.32 & 2.18 & 2.09 & 2.02 & 1.97 & 1.93 & 1.89 & 1.87 & 1.84 & 1.82 & 1.80 & 1.79 & 1.77 & 1.76 & 1.74 & 1.72 & 1.66 & 1.63 & 1.59 & 1.56 \\ 
  26 & 2.91 & 2.52 & 2.31 & 2.17 & 2.08 & 2.01 & 1.96 & 1.92 & 1.88 & 1.86 & 1.83 & 1.81 & 1.79 & 1.77 & 1.76 & 1.75 & 1.72 & 1.71 & 1.65 & 1.61 & 1.58 & 1.54 \\ 
  27 & 2.90 & 2.51 & 2.30 & 2.17 & 2.07 & 2.00 & 1.95 & 1.91 & 1.87 & 1.85 & 1.82 & 1.80 & 1.78 & 1.76 & 1.75 & 1.74 & 1.71 & 1.70 & 1.64 & 1.60 & 1.57 & 1.53 \\ 
  28 & 2.89 & 2.50 & 2.29 & 2.16 & 2.06 & 2.00 & 1.94 & 1.90 & 1.87 & 1.84 & 1.81 & 1.79 & 1.77 & 1.75 & 1.74 & 1.73 & 1.70 & 1.69 & 1.63 & 1.59 & 1.56 & 1.52 \\ 
  29 & 2.89 & 2.50 & 2.28 & 2.15 & 2.06 & 1.99 & 1.93 & 1.89 & 1.86 & 1.83 & 1.80 & 1.78 & 1.76 & 1.75 & 1.73 & 1.72 & 1.69 & 1.68 & 1.62 & 1.58 & 1.55 & 1.51 \\ 
  30 & 2.88 & 2.49 & 2.28 & 2.14 & 2.05 & 1.98 & 1.93 & 1.88 & 1.85 & 1.82 & 1.79 & 1.77 & 1.75 & 1.74 & 1.72 & 1.71 & 1.69 & 1.67 & 1.61 & 1.57 & 1.54 & 1.50 \\ 
  40 & 2.84 & 2.44 & 2.23 & 2.09 & 2.00 & 1.93 & 1.87 & 1.83 & 1.79 & 1.76 & 1.74 & 1.71 & 1.70 & 1.68 & 1.66 & 1.65 & 1.62 & 1.61 & 1.54 & 1.51 & 1.47 & 1.42 \\ 
  60 & 2.79 & 2.39 & 2.18 & 2.04 & 1.95 & 1.87 & 1.82 & 1.77 & 1.74 & 1.71 & 1.68 & 1.66 & 1.64 & 1.62 & 1.60 & 1.59 & 1.56 & 1.54 & 1.48 & 1.44 & 1.40 & 1.35 \\ 
  120 & 2.75 & 2.35 & 2.13 & 1.99 & 1.90 & 1.82 & 1.77 & 1.72 & 1.68 & 1.65 & 1.63 & 1.60 & 1.58 & 1.56 & 1.55 & 1.53 & 1.50 & 1.48 & 1.41 & 1.37 & 1.32 & 1.26 \\ 
   \hline
\end{tabular}
\end{table}