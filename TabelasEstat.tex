% !TeX spellcheck = pt_BR
% !TEX encoding = UTF-8 Unicode

% Copyright 2023 Felipe Queiroz. Todos os direitos reservados.Permitido o uso nos termos licença Creative Commons Atribuição-CompartilhaIgual 4.0 Internacional.

\documentclass[a4paper,12pt]{article}

\usepackage[authoryear]{natbib}
\usepackage[brazilian]{babel}
\usepackage[utf8]{inputenc}
\usepackage[T1]{fontenc}
\usepackage{lmodern}

\usepackage{graphicx}
\usepackage{amssymb}
\usepackage{amsmath}
\usepackage{amsthm}
\usepackage{lmodern}
\usepackage{bm, color}
\usepackage[pdftex]{hyperref}
\usepackage{framed}
\usepackage[all]{xy}
\usepackage{multicol}
\usepackage{multirow, tabularx}
\usepackage[top=30pt,bottom=40pt,left=40pt,right=40pt,headsep=0.4in]{geometry}
\usepackage{float}
 \usepackage{pdflscape}
 
 
 \usepackage{wrapfig}

\hypersetup{
hidelinks,
colorlinks,
urlcolor=[rgb]{0,0,.7},
linkcolor=[rgb]{.4,0,0},
citecolor=[rgb]{0,.3,0},
pdfstartview=FitH,
pdfpagemode=UseNone
}


\pagenumbering{gobble}


\begin{document}


\begin{minipage}[c]{.64\textwidth}

\begin{Huge}
Distribuição Normal
\end{Huge}

\vspace{2em}

A tabela fornece $\mathbb{P}(0 \leq Z \leq z)$ para $z>0$, em que $Z \sim \text{N}(0,1)$.

A primeira coluna mostra a parte inteira e primeira casa decimal de $z$.
A primeira linha mostra a segunda casa decimal de $z$.
\par\bigskip
\href{https://creativecommons.org/licenses/by-sa/4.0/deed.pt_BR}{\includegraphics[height=1em]{cc-by-sa.pdf}}
\href{https://rea-mecc.github.io}{REA-MECC}
\end{minipage}
\hfill
\begin{minipage}[c]{.35\textwidth}
\includegraphics[width=\linewidth]{plotnormal.pdf}
\end{minipage}

\input{tablenormal}


\newpage


\begin{landscape}

\begin{figure}[h]
\begin{minipage}[c]{1\textwidth}

\begin{Huge}
Distribuição $t$ de Student
\end{Huge}

\vspace{2em}

A tabela fornece o valor de $t$, tal que $\mathbb{P}(t_{(\nu)} \geq t) = p/2$, em que $\nu$ são os graus de liberdade.

Os valores de $p$ estão na primeira linha da tabela.
\par\bigskip
\href{https://creativecommons.org/licenses/by-sa/4.0/deed.pt_BR}{\includegraphics[height=1em]{cc-by-sa.pdf}}
\href{https://rea-mecc.github.io}{REA-MECC}
\end{minipage}
\hfill
\begin{minipage}[c]{.3\textwidth}
\includegraphics[width=\linewidth]{plott.pdf}
\end{minipage}
\end{figure}  

\input{tablet}

\end{landscape}


\newpage


\begin{landscape}

\begin{figure}[h]
\begin{minipage}[c]{1\textwidth}

\begin{Huge}
Distribuição $\chi^2$
\end{Huge}

\vspace{2em}

A tabela fornece o valor de $q$, tal que $\mathbb{P}(\chi^2_{(\nu)} \geq q) = p$, em que $\nu$ são os graus de liberdade.

Os valores de $p$ estão na primeira linha da tabela.
\par\bigskip
\href{https://creativecommons.org/licenses/by-sa/4.0/deed.pt_BR}{\includegraphics[height=1em]{cc-by-sa.pdf}}
\href{https://rea-mecc.github.io}{REA-MECC}
\end{minipage}
\hfill
\begin{minipage}[c]{.30\textwidth}
\includegraphics[width=\linewidth]{plotchi.pdf}
\end{minipage}
\end{figure}  

% Tabela gerada pelo R com pequenas modificações feitas manualmente
\begin{table}[H]
\centering
\scriptsize
\begin{tabular}{r|ccccc|ccccc|ccccc|ccccc}
  \hline
$\nu$ & 99\% & 98\% & 97.5\% & 95\% & 90\% & 80\% & 70\% & 60\% & 50\% & 40\% & 30\% & 20\% & 10\% & 5\% & 4\% & 2.5\% & 2\% & 1\% & 0.2\% & 0.1\% \\ 
  \hline
1 & 0.000 & 0.001 & 0.001 & 0.004 & 0.016 & 0.064 & 0.148 & 0.275 & 0.455 & 0.708 & 1.074 & 1.642 & 2.706 & 3.841 & 4.218 & 5.024 & 5.412 & 6.635 & 9.550 & 10.828 \\ 
  2 & 0.020 & 0.040 & 0.051 & 0.103 & 0.211 & 0.446 & 0.713 & 1.022 & 1.386 & 1.833 & 2.408 & 3.219 & 4.605 & 5.991 & 6.438 & 7.378 & 7.824 & 9.210 & 12.429 & 13.816 \\ 
  3 & 0.115 & 0.185 & 0.216 & 0.352 & 0.584 & 1.005 & 1.424 & 1.869 & 2.366 & 2.946 & 3.665 & 4.642 & 6.251 & 7.815 & 8.311 & 9.348 & 9.837 & 11.345 & 14.796 & 16.266 \\ 
  4 & 0.297 & 0.429 & 0.484 & 0.711 & 1.064 & 1.649 & 2.195 & 2.753 & 3.357 & 4.045 & 4.878 & 5.989 & 7.779 & 9.488 & 10.026 & 11.143 & 11.668 & 13.277 & 16.924 & 18.467 \\ 
  5 & 0.554 & 0.752 & 0.831 & 1.145 & 1.610 & 2.343 & 3.000 & 3.655 & 4.351 & 5.132 & 6.064 & 7.289 & 9.236 & 11.070 & 11.644 & 12.833 & 13.388 & 15.086 & 18.907 & 20.515 \\ 
  \hline
  6 & 0.872 & 1.134 & 1.237 & 1.635 & 2.204 & 3.070 & 3.828 & 4.570 & 5.348 & 6.211 & 7.231 & 8.558 & 10.645 & 12.592 & 13.198 & 14.449 & 15.033 & 16.812 & 20.791 & 22.458 \\ 
  7 & 1.239 & 1.564 & 1.690 & 2.167 & 2.833 & 3.822 & 4.671 & 5.493 & 6.346 & 7.283 & 8.383 & 9.803 & 12.017 & 14.067 & 14.703 & 16.013 & 16.622 & 18.475 & 22.601 & 24.322 \\ 
  8 & 1.646 & 2.032 & 2.180 & 2.733 & 3.490 & 4.594 & 5.527 & 6.423 & 7.344 & 8.351 & 9.524 & 11.030 & 13.362 & 15.507 & 16.171 & 17.535 & 18.168 & 20.090 & 24.352 & 26.124 \\ 
  9 & 2.088 & 2.532 & 2.700 & 3.325 & 4.168 & 5.380 & 6.393 & 7.357 & 8.343 & 9.414 & 10.656 & 12.242 & 14.684 & 16.919 & 17.608 & 19.023 & 19.679 & 21.666 & 26.056 & 27.877 \\ 
  10 & 2.558 & 3.059 & 3.247 & 3.940 & 4.865 & 6.179 & 7.267 & 8.295 & 9.342 & 10.473 & 11.781 & 13.442 & 15.987 & 18.307 & 19.021 & 20.483 & 21.161 & 23.209 & 27.722 & 29.588 \\ 
  \hline
  11 & 3.053 & 3.609 & 3.816 & 4.575 & 5.578 & 6.989 & 8.148 & 9.237 & 10.341 & 11.530 & 12.899 & 14.631 & 17.275 & 19.675 & 20.412 & 21.920 & 22.618 & 24.725 & 29.354 & 31.264 \\ 
  12 & 3.571 & 4.178 & 4.404 & 5.226 & 6.304 & 7.807 & 9.034 & 10.182 & 11.340 & 12.584 & 14.011 & 15.812 & 18.549 & 21.026 & 21.785 & 23.337 & 24.054 & 26.217 & 30.957 & 32.909 \\ 
  13 & 4.107 & 4.765 & 5.009 & 5.892 & 7.042 & 8.634 & 9.926 & 11.129 & 12.340 & 13.636 & 15.119 & 16.985 & 19.812 & 22.362 & 23.142 & 24.736 & 25.472 & 27.688 & 32.535 & 34.528 \\ 
  14 & 4.660 & 5.368 & 5.629 & 6.571 & 7.790 & 9.467 & 10.821 & 12.078 & 13.339 & 14.685 & 16.222 & 18.151 & 21.064 & 23.685 & 24.485 & 26.119 & 26.873 & 29.141 & 34.091 & 36.123 \\ 
  15 & 5.229 & 5.985 & 6.262 & 7.261 & 8.547 & 10.307 & 11.721 & 13.030 & 14.339 & 15.733 & 17.322 & 19.311 & 22.307 & 24.996 & 25.816 & 27.488 & 28.259 & 30.578 & 35.628 & 37.697 \\ 
  \hline
  16 & 5.812 & 6.614 & 6.908 & 7.962 & 9.312 & 11.152 & 12.624 & 13.983 & 15.338 & 16.780 & 18.418 & 20.465 & 23.542 & 26.296 & 27.136 & 28.845 & 29.633 & 32.000 & 37.146 & 39.252 \\ 
  17 & 6.408 & 7.255 & 7.564 & 8.672 & 10.085 & 12.002 & 13.531 & 14.937 & 16.338 & 17.824 & 19.511 & 21.615 & 24.769 & 27.587 & 28.445 & 30.191 & 30.995 & 33.409 & 38.648 & 40.790 \\ 
  18 & 7.015 & 7.906 & 8.231 & 9.390 & 10.865 & 12.857 & 14.440 & 15.893 & 17.338 & 18.868 & 20.601 & 22.760 & 25.989 & 28.869 & 29.745 & 31.526 & 32.346 & 34.805 & 40.136 & 42.312 \\ 
  19 & 7.633 & 8.567 & 8.907 & 10.117 & 11.651 & 13.716 & 15.352 & 16.850 & 18.338 & 19.910 & 21.689 & 23.900 & 27.204 & 30.144 & 31.037 & 32.852 & 33.687 & 36.191 & 41.610 & 43.820 \\ 
  20 & 8.260 & 9.237 & 9.591 & 10.851 & 12.443 & 14.578 & 16.266 & 17.809 & 19.337 & 20.951 & 22.775 & 25.038 & 28.412 & 31.410 & 32.321 & 34.170 & 35.020 & 37.566 & 43.072 & 45.315 \\ 
  \hline
  21 & 8.897 & 9.915 & 10.283 & 11.591 & 13.240 & 15.445 & 17.182 & 18.768 & 20.337 & 21.991 & 23.858 & 26.171 & 29.615 & 32.671 & 33.597 & 35.479 & 36.343 & 38.932 & 44.522 & 46.797 \\ 
  22 & 9.542 & 10.600 & 10.982 & 12.338 & 14.041 & 16.314 & 18.101 & 19.729 & 21.337 & 23.031 & 24.939 & 27.301 & 30.813 & 33.924 & 34.867 & 36.781 & 37.659 & 40.289 & 45.962 & 48.268 \\ 
  23 & 10.196 & 11.293 & 11.689 & 13.091 & 14.848 & 17.187 & 19.021 & 20.690 & 22.337 & 24.069 & 26.018 & 28.429 & 32.007 & 35.172 & 36.131 & 38.076 & 38.968 & 41.638 & 47.391 & 49.728 \\ 
  24 & 10.856 & 11.992 & 12.401 & 13.848 & 15.659 & 18.062 & 19.943 & 21.652 & 23.337 & 25.106 & 27.096 & 29.553 & 33.196 & 36.415 & 37.389 & 39.364 & 40.270 & 42.980 & 48.812 & 51.179 \\ 
  25 & 11.524 & 12.697 & 13.120 & 14.611 & 16.473 & 18.940 & 20.867 & 22.616 & 24.337 & 26.143 & 28.172 & 30.675 & 34.382 & 37.652 & 38.642 & 40.646 & 41.566 & 44.314 & 50.223 & 52.620 \\ 
  \hline
  26 & 12.198 & 13.409 & 13.844 & 15.379 & 17.292 & 19.820 & 21.792 & 23.579 & 25.336 & 27.179 & 29.246 & 31.795 & 35.563 & 38.885 & 39.889 & 41.923 & 42.856 & 45.642 & 51.627 & 54.052 \\ 
  27 & 12.879 & 14.125 & 14.573 & 16.151 & 18.114 & 20.703 & 22.719 & 24.544 & 26.336 & 28.214 & 30.319 & 32.912 & 36.741 & 40.113 & 41.132 & 43.195 & 44.140 & 46.963 & 53.023 & 55.476 \\ 
  28 & 13.565 & 14.847 & 15.308 & 16.928 & 18.939 & 21.588 & 23.647 & 25.509 & 27.336 & 29.249 & 31.391 & 34.027 & 37.916 & 41.337 & 42.370 & 44.461 & 45.419 & 48.278 & 54.411 & 56.892 \\ 
  29 & 14.256 & 15.574 & 16.047 & 17.708 & 19.768 & 22.475 & 24.577 & 26.475 & 28.336 & 30.283 & 32.461 & 35.139 & 39.087 & 42.557 & 43.604 & 45.722 & 46.693 & 49.588 & 55.792 & 58.301 \\ 
  30 & 14.953 & 16.306 & 16.791 & 18.493 & 20.599 & 23.364 & 25.508 & 27.442 & 29.336 & 31.316 & 33.530 & 36.250 & 40.256 & 43.773 & 44.834 & 46.979 & 47.962 & 50.892 & 57.167 & 59.703 \\ 
  \hline
  35 & 18.509 & 20.027 & 20.569 & 22.465 & 24.797 & 27.836 & 30.178 & 32.282 & 34.336 & 36.475 & 38.859 & 41.778 & 46.059 & 49.802 & 50.928 & 53.203 & 54.244 & 57.342 & 63.955 & 66.619 \\ 
  40 & 22.164 & 23.838 & 24.433 & 26.509 & 29.051 & 32.345 & 34.872 & 37.134 & 39.335 & 41.622 & 44.165 & 47.269 & 51.805 & 55.758 & 56.946 & 59.342 & 60.436 & 63.691 & 70.618 & 73.402 \\ 
  45 & 25.901 & 27.720 & 28.366 & 30.612 & 33.350 & 36.884 & 39.585 & 41.995 & 44.335 & 46.761 & 49.452 & 52.729 & 57.505 & 61.656 & 62.901 & 65.410 & 66.555 & 69.957 & 77.179 & 80.077 \\ 
  50 & 29.707 & 31.664 & 32.357 & 34.764 & 37.689 & 41.449 & 44.313 & 46.864 & 49.335 & 51.892 & 54.723 & 58.164 & 63.167 & 67.505 & 68.804 & 71.420 & 72.613 & 76.154 & 83.657 & 86.661 \\ 
   \hline
\end{tabular}
\end{table}


\end{landscape}


\newpage


\renewcommand{\arraystretch}{1.12}
\setlength\tabcolsep{.45em}


\begin{landscape}

\begin{minipage}[l]{1\textwidth}
\begin{Huge}
Distribuição F de Snedecor ($10\%$)
\end{Huge}
\par
\vspace{2em}
A tabela fornece o valor de $q$, tal que $\mathbb{P}(F_{(\nu_1, \nu_2)} \geq q) = 0.1$, em que $\nu_1$ são os graus de liberdade do numerador e $\nu_2$ os graus de liberdade do denominador. 
\par\bigskip
\href{https://creativecommons.org/licenses/by-sa/4.0/deed.pt_BR}{\includegraphics[height=1em]{cc-by-sa.pdf}}
\href{https://rea-mecc.github.io}{REA-MECC}
\end{minipage}
\hfill
\begin{minipage}[c]{.35\textwidth}
\includegraphics[height=.9\linewidth]{plotf01.pdf}
\end{minipage}

% Tabela gerada pelo R com pequenas modificações feitas manualmente
\begin{table}[H]
\centering
\scriptsize
\begin{tabular}{r|rrrrrrrrrrrrrrrrrrrrrrrrrrrrr}
  \hline
 & \multicolumn{29}{c}{$\nu_1$} \\
  \hline
$\nu_2$ & 5 & 6 & 7 & 8 & 9 & 10 & 11 & 12 & 13 & 14 & 15 & 16 & 17 & 18 & 19 & 20 & 21 & 22 & 23 & 24 & 25 & 26 & 27 & 28 & 29 & 30 & 40 & 60 & 120 \\ 
  \hline
5 & 3.45 & 3.40 & 3.37 & 3.34 & 3.32 & 3.30 & 3.28 & 3.27 & 3.26 & 3.25 & 3.24 & 3.23 & 3.22 & 3.22 & 3.21 & 3.21 & 3.20 & 3.20 & 3.19 & 3.19 & 3.19 & 3.18 & 3.18 & 3.18 & 3.18 & 3.17 & 3.16 & 3.14 & 3.12 \\ 
  6 & 3.11 & 3.05 & 3.01 & 2.98 & 2.96 & 2.94 & 2.92 & 2.90 & 2.89 & 2.88 & 2.87 & 2.86 & 2.85 & 2.85 & 2.84 & 2.84 & 2.83 & 2.83 & 2.82 & 2.82 & 2.81 & 2.81 & 2.81 & 2.81 & 2.80 & 2.80 & 2.78 & 2.76 & 2.74 \\ 
  7 & 2.88 & 2.83 & 2.78 & 2.75 & 2.72 & 2.70 & 2.68 & 2.67 & 2.65 & 2.64 & 2.63 & 2.62 & 2.61 & 2.61 & 2.60 & 2.59 & 2.59 & 2.58 & 2.58 & 2.58 & 2.57 & 2.57 & 2.56 & 2.56 & 2.56 & 2.56 & 2.54 & 2.51 & 2.49 \\ 
  8 & 2.73 & 2.67 & 2.62 & 2.59 & 2.56 & 2.54 & 2.52 & 2.50 & 2.49 & 2.48 & 2.46 & 2.45 & 2.45 & 2.44 & 2.43 & 2.42 & 2.42 & 2.41 & 2.41 & 2.40 & 2.40 & 2.40 & 2.39 & 2.39 & 2.39 & 2.38 & 2.36 & 2.34 & 2.32 \\ 
  9 & 2.61 & 2.55 & 2.51 & 2.47 & 2.44 & 2.42 & 2.40 & 2.38 & 2.36 & 2.35 & 2.34 & 2.33 & 2.32 & 2.31 & 2.30 & 2.30 & 2.29 & 2.29 & 2.28 & 2.28 & 2.27 & 2.27 & 2.26 & 2.26 & 2.26 & 2.25 & 2.23 & 2.21 & 2.18 \\ 
  10 & 2.52 & 2.46 & 2.41 & 2.38 & 2.35 & 2.32 & 2.30 & 2.28 & 2.27 & 2.26 & 2.24 & 2.23 & 2.22 & 2.22 & 2.21 & 2.20 & 2.19 & 2.19 & 2.18 & 2.18 & 2.17 & 2.17 & 2.17 & 2.16 & 2.16 & 2.16 & 2.13 & 2.11 & 2.08 \\ 
  11 & 2.45 & 2.39 & 2.34 & 2.30 & 2.27 & 2.25 & 2.23 & 2.21 & 2.19 & 2.18 & 2.17 & 2.16 & 2.15 & 2.14 & 2.13 & 2.12 & 2.12 & 2.11 & 2.11 & 2.10 & 2.10 & 2.09 & 2.09 & 2.08 & 2.08 & 2.08 & 2.05 & 2.03 & 2.00 \\ 
  12 & 2.39 & 2.33 & 2.28 & 2.24 & 2.21 & 2.19 & 2.17 & 2.15 & 2.13 & 2.12 & 2.10 & 2.09 & 2.08 & 2.08 & 2.07 & 2.06 & 2.05 & 2.05 & 2.04 & 2.04 & 2.03 & 2.03 & 2.02 & 2.02 & 2.01 & 2.01 & 1.99 & 1.96 & 1.93 \\ 
  13 & 2.35 & 2.28 & 2.23 & 2.20 & 2.16 & 2.14 & 2.12 & 2.10 & 2.08 & 2.07 & 2.05 & 2.04 & 2.03 & 2.02 & 2.01 & 2.01 & 2.00 & 1.99 & 1.99 & 1.98 & 1.98 & 1.97 & 1.97 & 1.96 & 1.96 & 1.96 & 1.93 & 1.90 & 1.88 \\ 
  14 & 2.31 & 2.24 & 2.19 & 2.15 & 2.12 & 2.10 & 2.07 & 2.05 & 2.04 & 2.02 & 2.01 & 2.00 & 1.99 & 1.98 & 1.97 & 1.96 & 1.96 & 1.95 & 1.94 & 1.94 & 1.93 & 1.93 & 1.92 & 1.92 & 1.92 & 1.91 & 1.89 & 1.86 & 1.83 \\ 
  15 & 2.27 & 2.21 & 2.16 & 2.12 & 2.09 & 2.06 & 2.04 & 2.02 & 2.00 & 1.99 & 1.97 & 1.96 & 1.95 & 1.94 & 1.93 & 1.92 & 1.92 & 1.91 & 1.90 & 1.90 & 1.89 & 1.89 & 1.88 & 1.88 & 1.88 & 1.87 & 1.85 & 1.82 & 1.79 \\ 
  16 & 2.24 & 2.18 & 2.13 & 2.09 & 2.06 & 2.03 & 2.01 & 1.99 & 1.97 & 1.95 & 1.94 & 1.93 & 1.92 & 1.91 & 1.90 & 1.89 & 1.88 & 1.88 & 1.87 & 1.87 & 1.86 & 1.86 & 1.85 & 1.85 & 1.84 & 1.84 & 1.81 & 1.78 & 1.75 \\ 
  17 & 2.22 & 2.15 & 2.10 & 2.06 & 2.03 & 2.00 & 1.98 & 1.96 & 1.94 & 1.93 & 1.91 & 1.90 & 1.89 & 1.88 & 1.87 & 1.86 & 1.86 & 1.85 & 1.84 & 1.84 & 1.83 & 1.83 & 1.82 & 1.82 & 1.81 & 1.81 & 1.78 & 1.75 & 1.72 \\ 
  18 & 2.20 & 2.13 & 2.08 & 2.04 & 2.00 & 1.98 & 1.95 & 1.93 & 1.92 & 1.90 & 1.89 & 1.87 & 1.86 & 1.85 & 1.84 & 1.84 & 1.83 & 1.82 & 1.82 & 1.81 & 1.80 & 1.80 & 1.80 & 1.79 & 1.79 & 1.78 & 1.75 & 1.72 & 1.69 \\ 
  19 & 2.18 & 2.11 & 2.06 & 2.02 & 1.98 & 1.96 & 1.93 & 1.91 & 1.89 & 1.88 & 1.86 & 1.85 & 1.84 & 1.83 & 1.82 & 1.81 & 1.81 & 1.80 & 1.79 & 1.79 & 1.78 & 1.78 & 1.77 & 1.77 & 1.76 & 1.76 & 1.73 & 1.70 & 1.67 \\ 
  20 & 2.16 & 2.09 & 2.04 & 2.00 & 1.96 & 1.94 & 1.91 & 1.89 & 1.87 & 1.86 & 1.84 & 1.83 & 1.82 & 1.81 & 1.80 & 1.79 & 1.79 & 1.78 & 1.77 & 1.77 & 1.76 & 1.76 & 1.75 & 1.75 & 1.74 & 1.74 & 1.71 & 1.68 & 1.64 \\ 
  21 & 2.14 & 2.08 & 2.02 & 1.98 & 1.95 & 1.92 & 1.90 & 1.87 & 1.86 & 1.84 & 1.83 & 1.81 & 1.80 & 1.79 & 1.78 & 1.78 & 1.77 & 1.76 & 1.75 & 1.75 & 1.74 & 1.74 & 1.73 & 1.73 & 1.72 & 1.72 & 1.69 & 1.66 & 1.62 \\ 
  22 & 2.13 & 2.06 & 2.01 & 1.97 & 1.93 & 1.90 & 1.88 & 1.86 & 1.84 & 1.83 & 1.81 & 1.80 & 1.79 & 1.78 & 1.77 & 1.76 & 1.75 & 1.74 & 1.74 & 1.73 & 1.73 & 1.72 & 1.72 & 1.71 & 1.71 & 1.70 & 1.67 & 1.64 & 1.60 \\ 
  23 & 2.11 & 2.05 & 1.99 & 1.95 & 1.92 & 1.89 & 1.87 & 1.84 & 1.83 & 1.81 & 1.80 & 1.78 & 1.77 & 1.76 & 1.75 & 1.74 & 1.74 & 1.73 & 1.72 & 1.72 & 1.71 & 1.70 & 1.70 & 1.69 & 1.69 & 1.69 & 1.66 & 1.62 & 1.59 \\ 
  24 & 2.10 & 2.04 & 1.98 & 1.94 & 1.91 & 1.88 & 1.85 & 1.83 & 1.81 & 1.80 & 1.78 & 1.77 & 1.76 & 1.75 & 1.74 & 1.73 & 1.72 & 1.71 & 1.71 & 1.70 & 1.70 & 1.69 & 1.69 & 1.68 & 1.68 & 1.67 & 1.64 & 1.61 & 1.57 \\ 
  25 & 2.09 & 2.02 & 1.97 & 1.93 & 1.89 & 1.87 & 1.84 & 1.82 & 1.80 & 1.79 & 1.77 & 1.76 & 1.75 & 1.74 & 1.73 & 1.72 & 1.71 & 1.70 & 1.70 & 1.69 & 1.68 & 1.68 & 1.67 & 1.67 & 1.66 & 1.66 & 1.63 & 1.59 & 1.56 \\ 
  26 & 2.08 & 2.01 & 1.96 & 1.92 & 1.88 & 1.86 & 1.83 & 1.81 & 1.79 & 1.77 & 1.76 & 1.75 & 1.73 & 1.72 & 1.71 & 1.71 & 1.70 & 1.69 & 1.68 & 1.68 & 1.67 & 1.67 & 1.66 & 1.66 & 1.65 & 1.65 & 1.61 & 1.58 & 1.54 \\ 
  27 & 2.07 & 2.00 & 1.95 & 1.91 & 1.87 & 1.85 & 1.82 & 1.80 & 1.78 & 1.76 & 1.75 & 1.74 & 1.72 & 1.71 & 1.70 & 1.70 & 1.69 & 1.68 & 1.67 & 1.67 & 1.66 & 1.65 & 1.65 & 1.64 & 1.64 & 1.64 & 1.60 & 1.57 & 1.53 \\ 
  28 & 2.06 & 2.00 & 1.94 & 1.90 & 1.87 & 1.84 & 1.81 & 1.79 & 1.77 & 1.75 & 1.74 & 1.73 & 1.71 & 1.70 & 1.69 & 1.69 & 1.68 & 1.67 & 1.66 & 1.66 & 1.65 & 1.64 & 1.64 & 1.63 & 1.63 & 1.63 & 1.59 & 1.56 & 1.52 \\ 
  29 & 2.06 & 1.99 & 1.93 & 1.89 & 1.86 & 1.83 & 1.80 & 1.78 & 1.76 & 1.75 & 1.73 & 1.72 & 1.71 & 1.69 & 1.68 & 1.68 & 1.67 & 1.66 & 1.65 & 1.65 & 1.64 & 1.63 & 1.63 & 1.62 & 1.62 & 1.62 & 1.58 & 1.55 & 1.51 \\ 
  30 & 2.05 & 1.98 & 1.93 & 1.88 & 1.85 & 1.82 & 1.79 & 1.77 & 1.75 & 1.74 & 1.72 & 1.71 & 1.70 & 1.69 & 1.68 & 1.67 & 1.66 & 1.65 & 1.64 & 1.64 & 1.63 & 1.63 & 1.62 & 1.62 & 1.61 & 1.61 & 1.57 & 1.54 & 1.50 \\ 
  40 & 2.00 & 1.93 & 1.87 & 1.83 & 1.79 & 1.76 & 1.74 & 1.71 & 1.70 & 1.68 & 1.66 & 1.65 & 1.64 & 1.62 & 1.61 & 1.61 & 1.60 & 1.59 & 1.58 & 1.57 & 1.57 & 1.56 & 1.56 & 1.55 & 1.55 & 1.54 & 1.51 & 1.47 & 1.42 \\ 
  60 & 1.95 & 1.87 & 1.82 & 1.77 & 1.74 & 1.71 & 1.68 & 1.66 & 1.64 & 1.62 & 1.60 & 1.59 & 1.58 & 1.56 & 1.55 & 1.54 & 1.53 & 1.53 & 1.52 & 1.51 & 1.50 & 1.50 & 1.49 & 1.49 & 1.48 & 1.48 & 1.44 & 1.40 & 1.35 \\ 
  120 & 1.90 & 1.82 & 1.77 & 1.72 & 1.68 & 1.65 & 1.63 & 1.60 & 1.58 & 1.56 & 1.55 & 1.53 & 1.52 & 1.50 & 1.49 & 1.48 & 1.47 & 1.46 & 1.46 & 1.45 & 1.44 & 1.43 & 1.43 & 1.42 & 1.41 & 1.41 & 1.37 & 1.32 & 1.26 \\ 
   \hline
\end{tabular}
\end{table}


\end{landscape}


\newpage


\begin{landscape}

\begin{minipage}[l]{1\textwidth}
\begin{Huge}
Distribuição F de Snedecor ($5\%$)
\end{Huge}
\par
\vspace{2em}
A tabela fornece o valor de $q$, tal que $\mathbb{P}(F_{(\nu_1, \nu_2)} \geq q) = 0.05$, em que $\nu_1$ são os graus de liberdade do numerador e $\nu_2$ os graus de liberdade do denominador. 
\par\bigskip
\href{https://creativecommons.org/licenses/by-sa/4.0/deed.pt_BR}{\includegraphics[height=1em]{cc-by-sa.pdf}}
\href{https://rea-mecc.github.io}{REA-MECC}
\end{minipage}
\hfill
\begin{minipage}[c]{.35\textwidth}
\includegraphics[height=.9\linewidth]{plotf005.pdf}
\end{minipage}

% Tabela gerada pelo R com pequenas modificações feitas manualmente
\begin{table}[H]
\centering
\scriptsize
\begin{tabular}{rcccccccccccccccccccccc}
  \hline
 & \multicolumn{22}{c}{$\nu_1$} \\

$\nu_2$ & 1 & 2 & 3 & 4 & 5 & 6 & 7 & 8 & 9 & 10 & 11 & 12 & 13 & 14 & 15 & 16 & 18 & 20 & 30 & 40 & 60 & 120 \\ 
  \hline
2 & 18.51 & 19.00 & 19.16 & 19.25 & 19.30 & 19.33 & 19.35 & 19.37 & 19.38 & 19.40 & 19.40 & 19.41 & 19.42 & 19.42 & 19.43 & 19.43 & 19.44 & 19.45 & 19.46 & 19.47 & 19.48 & 19.49 \\ 
  3 & 10.13 & 9.55 & 9.28 & 9.12 & 9.01 & 8.94 & 8.89 & 8.85 & 8.81 & 8.79 & 8.76 & 8.74 & 8.73 & 8.71 & 8.70 & 8.69 & 8.67 & 8.66 & 8.62 & 8.59 & 8.57 & 8.55 \\ 
  4 & 7.71 & 6.94 & 6.59 & 6.39 & 6.26 & 6.16 & 6.09 & 6.04 & 6.00 & 5.96 & 5.94 & 5.91 & 5.89 & 5.87 & 5.86 & 5.84 & 5.82 & 5.80 & 5.75 & 5.72 & 5.69 & 5.66 \\ 
  5 & 6.61 & 5.79 & 5.41 & 5.19 & 5.05 & 4.95 & 4.88 & 4.82 & 4.77 & 4.74 & 4.70 & 4.68 & 4.66 & 4.64 & 4.62 & 4.60 & 4.58 & 4.56 & 4.50 & 4.46 & 4.43 & 4.40 \\ 
  6 & 5.99 & 5.14 & 4.76 & 4.53 & 4.39 & 4.28 & 4.21 & 4.15 & 4.10 & 4.06 & 4.03 & 4.00 & 3.98 & 3.96 & 3.94 & 3.92 & 3.90 & 3.87 & 3.81 & 3.77 & 3.74 & 3.70 \\ 
  7 & 5.59 & 4.74 & 4.35 & 4.12 & 3.97 & 3.87 & 3.79 & 3.73 & 3.68 & 3.64 & 3.60 & 3.57 & 3.55 & 3.53 & 3.51 & 3.49 & 3.47 & 3.44 & 3.38 & 3.34 & 3.30 & 3.27 \\ 
  8 & 5.32 & 4.46 & 4.07 & 3.84 & 3.69 & 3.58 & 3.50 & 3.44 & 3.39 & 3.35 & 3.31 & 3.28 & 3.26 & 3.24 & 3.22 & 3.20 & 3.17 & 3.15 & 3.08 & 3.04 & 3.01 & 2.97 \\ 
  9 & 5.12 & 4.26 & 3.86 & 3.63 & 3.48 & 3.37 & 3.29 & 3.23 & 3.18 & 3.14 & 3.10 & 3.07 & 3.05 & 3.03 & 3.01 & 2.99 & 2.96 & 2.94 & 2.86 & 2.83 & 2.79 & 2.75 \\ 
  10 & 4.96 & 4.10 & 3.71 & 3.48 & 3.33 & 3.22 & 3.14 & 3.07 & 3.02 & 2.98 & 2.94 & 2.91 & 2.89 & 2.86 & 2.85 & 2.83 & 2.80 & 2.77 & 2.70 & 2.66 & 2.62 & 2.58 \\ 
  11 & 4.84 & 3.98 & 3.59 & 3.36 & 3.20 & 3.09 & 3.01 & 2.95 & 2.90 & 2.85 & 2.82 & 2.79 & 2.76 & 2.74 & 2.72 & 2.70 & 2.67 & 2.65 & 2.57 & 2.53 & 2.49 & 2.45 \\ 
  12 & 4.75 & 3.89 & 3.49 & 3.26 & 3.11 & 3.00 & 2.91 & 2.85 & 2.80 & 2.75 & 2.72 & 2.69 & 2.66 & 2.64 & 2.62 & 2.60 & 2.57 & 2.54 & 2.47 & 2.43 & 2.38 & 2.34 \\ 
  13 & 4.67 & 3.81 & 3.41 & 3.18 & 3.03 & 2.92 & 2.83 & 2.77 & 2.71 & 2.67 & 2.63 & 2.60 & 2.58 & 2.55 & 2.53 & 2.51 & 2.48 & 2.46 & 2.38 & 2.34 & 2.30 & 2.25 \\ 
  14 & 4.60 & 3.74 & 3.34 & 3.11 & 2.96 & 2.85 & 2.76 & 2.70 & 2.65 & 2.60 & 2.57 & 2.53 & 2.51 & 2.48 & 2.46 & 2.44 & 2.41 & 2.39 & 2.31 & 2.27 & 2.22 & 2.18 \\ 
  15 & 4.54 & 3.68 & 3.29 & 3.06 & 2.90 & 2.79 & 2.71 & 2.64 & 2.59 & 2.54 & 2.51 & 2.48 & 2.45 & 2.42 & 2.40 & 2.38 & 2.35 & 2.33 & 2.25 & 2.20 & 2.16 & 2.11 \\ 
  16 & 4.49 & 3.63 & 3.24 & 3.01 & 2.85 & 2.74 & 2.66 & 2.59 & 2.54 & 2.49 & 2.46 & 2.42 & 2.40 & 2.37 & 2.35 & 2.33 & 2.30 & 2.28 & 2.19 & 2.15 & 2.11 & 2.06 \\ 
  17 & 4.45 & 3.59 & 3.20 & 2.96 & 2.81 & 2.70 & 2.61 & 2.55 & 2.49 & 2.45 & 2.41 & 2.38 & 2.35 & 2.33 & 2.31 & 2.29 & 2.26 & 2.23 & 2.15 & 2.10 & 2.06 & 2.01 \\ 
  18 & 4.41 & 3.55 & 3.16 & 2.93 & 2.77 & 2.66 & 2.58 & 2.51 & 2.46 & 2.41 & 2.37 & 2.34 & 2.31 & 2.29 & 2.27 & 2.25 & 2.22 & 2.19 & 2.11 & 2.06 & 2.02 & 1.97 \\ 
  19 & 4.38 & 3.52 & 3.13 & 2.90 & 2.74 & 2.63 & 2.54 & 2.48 & 2.42 & 2.38 & 2.34 & 2.31 & 2.28 & 2.26 & 2.23 & 2.21 & 2.18 & 2.16 & 2.07 & 2.03 & 1.98 & 1.93 \\ 
  20 & 4.35 & 3.49 & 3.10 & 2.87 & 2.71 & 2.60 & 2.51 & 2.45 & 2.39 & 2.35 & 2.31 & 2.28 & 2.25 & 2.22 & 2.20 & 2.18 & 2.15 & 2.12 & 2.04 & 1.99 & 1.95 & 1.90 \\ 
  21 & 4.32 & 3.47 & 3.07 & 2.84 & 2.68 & 2.57 & 2.49 & 2.42 & 2.37 & 2.32 & 2.28 & 2.25 & 2.22 & 2.20 & 2.18 & 2.16 & 2.12 & 2.10 & 2.01 & 1.96 & 1.92 & 1.87 \\ 
  22 & 4.30 & 3.44 & 3.05 & 2.82 & 2.66 & 2.55 & 2.46 & 2.40 & 2.34 & 2.30 & 2.26 & 2.23 & 2.20 & 2.17 & 2.15 & 2.13 & 2.10 & 2.07 & 1.98 & 1.94 & 1.89 & 1.84 \\ 
  23 & 4.28 & 3.42 & 3.03 & 2.80 & 2.64 & 2.53 & 2.44 & 2.37 & 2.32 & 2.27 & 2.24 & 2.20 & 2.18 & 2.15 & 2.13 & 2.11 & 2.08 & 2.05 & 1.96 & 1.91 & 1.86 & 1.81 \\ 
  24 & 4.26 & 3.40 & 3.01 & 2.78 & 2.62 & 2.51 & 2.42 & 2.36 & 2.30 & 2.25 & 2.22 & 2.18 & 2.15 & 2.13 & 2.11 & 2.09 & 2.05 & 2.03 & 1.94 & 1.89 & 1.84 & 1.79 \\ 
  25 & 4.24 & 3.39 & 2.99 & 2.76 & 2.60 & 2.49 & 2.40 & 2.34 & 2.28 & 2.24 & 2.20 & 2.16 & 2.14 & 2.11 & 2.09 & 2.07 & 2.04 & 2.01 & 1.92 & 1.87 & 1.82 & 1.77 \\ 
  26 & 4.23 & 3.37 & 2.98 & 2.74 & 2.59 & 2.47 & 2.39 & 2.32 & 2.27 & 2.22 & 2.18 & 2.15 & 2.12 & 2.09 & 2.07 & 2.05 & 2.02 & 1.99 & 1.90 & 1.85 & 1.80 & 1.75 \\ 
  27 & 4.21 & 3.35 & 2.96 & 2.73 & 2.57 & 2.46 & 2.37 & 2.31 & 2.25 & 2.20 & 2.17 & 2.13 & 2.10 & 2.08 & 2.06 & 2.04 & 2.00 & 1.97 & 1.88 & 1.84 & 1.79 & 1.73 \\ 
  28 & 4.20 & 3.34 & 2.95 & 2.71 & 2.56 & 2.45 & 2.36 & 2.29 & 2.24 & 2.19 & 2.15 & 2.12 & 2.09 & 2.06 & 2.04 & 2.02 & 1.99 & 1.96 & 1.87 & 1.82 & 1.77 & 1.71 \\ 
  29 & 4.18 & 3.33 & 2.93 & 2.70 & 2.55 & 2.43 & 2.35 & 2.28 & 2.22 & 2.18 & 2.14 & 2.10 & 2.08 & 2.05 & 2.03 & 2.01 & 1.97 & 1.94 & 1.85 & 1.81 & 1.75 & 1.70 \\ 
  30 & 4.17 & 3.32 & 2.92 & 2.69 & 2.53 & 2.42 & 2.33 & 2.27 & 2.21 & 2.16 & 2.13 & 2.09 & 2.06 & 2.04 & 2.01 & 1.99 & 1.96 & 1.93 & 1.84 & 1.79 & 1.74 & 1.68 \\ 
  40 & 4.08 & 3.23 & 2.84 & 2.61 & 2.45 & 2.34 & 2.25 & 2.18 & 2.12 & 2.08 & 2.04 & 2.00 & 1.97 & 1.95 & 1.92 & 1.90 & 1.87 & 1.84 & 1.74 & 1.69 & 1.64 & 1.58 \\ 
  60 & 4.00 & 3.15 & 2.76 & 2.53 & 2.37 & 2.25 & 2.17 & 2.10 & 2.04 & 1.99 & 1.95 & 1.92 & 1.89 & 1.86 & 1.84 & 1.82 & 1.78 & 1.75 & 1.65 & 1.59 & 1.53 & 1.47 \\ 
  120 & 3.92 & 3.07 & 2.68 & 2.45 & 2.29 & 2.18 & 2.09 & 2.02 & 1.96 & 1.91 & 1.87 & 1.83 & 1.80 & 1.78 & 1.75 & 1.73 & 1.69 & 1.66 & 1.55 & 1.50 & 1.43 & 1.35 \\ 
   \hline
\end{tabular}
\end{table}


\end{landscape}


\newpage


\begin{landscape}

\begin{minipage}[l]{1\textwidth}
\begin{Huge}
Distribuição F de Snedecor ($2.5\%$)
\end{Huge}
\par
\vspace{2em}
A tabela fornece o valor de $q$, tal que $\mathbb{P}(F_{(\nu_1, \nu_2)} \geq q) = 0.025$, em que $\nu_1$ são os graus de liberdade do numerador e $\nu_2$ os graus de liberdade do denominador. 
\par\bigskip
\href{https://creativecommons.org/licenses/by-sa/4.0/deed.pt_BR}{\includegraphics[height=1em]{cc-by-sa.pdf}}
\href{https://rea-mecc.github.io}{REA-MECC}
\end{minipage}
\hfill
\begin{minipage}[c]{.35\textwidth}
\includegraphics[height=.9\linewidth]{plotf0025.pdf}
\end{minipage}

% Tabela gerada pelo R com pequenas modificações feitas manualmente
\begin{table}[H]
\centering
\scriptsize
\begin{tabular}{rcccccccccccccccccccccc}
  \hline
 & \multicolumn{22}{c}{$\nu_1$} \\

$\nu_2$ & 1 & 2 & 3 & 4 & 5 & 6 & 7 & 8 & 9 & 10 & 11 & 12 & 13 & 14 & 15 & 16 & 18 & 20 & 30 & 40 & 60 & 120 \\ 
  \hline
2 & 38.51 & 39.00 & 39.17 & 39.25 & 39.30 & 39.33 & 39.36 & 39.37 & 39.39 & 39.40 & 39.41 & 39.41 & 39.42 & 39.43 & 39.43 & 39.44 & 39.44 & 39.45 & 39.46 & 39.47 & 39.48 & 39.49 \\ 
  3 & 17.44 & 16.04 & 15.44 & 15.10 & 14.88 & 14.73 & 14.62 & 14.54 & 14.47 & 14.42 & 14.37 & 14.34 & 14.30 & 14.28 & 14.25 & 14.23 & 14.20 & 14.17 & 14.08 & 14.04 & 13.99 & 13.95 \\ 
  4 & 12.22 & 10.65 & 9.98 & 9.60 & 9.36 & 9.20 & 9.07 & 8.98 & 8.90 & 8.84 & 8.79 & 8.75 & 8.71 & 8.68 & 8.66 & 8.63 & 8.59 & 8.56 & 8.46 & 8.41 & 8.36 & 8.31 \\ 
  5 & 10.01 & 8.43 & 7.76 & 7.39 & 7.15 & 6.98 & 6.85 & 6.76 & 6.68 & 6.62 & 6.57 & 6.52 & 6.49 & 6.46 & 6.43 & 6.40 & 6.36 & 6.33 & 6.23 & 6.18 & 6.12 & 6.07 \\ 
  6 & 8.81 & 7.26 & 6.60 & 6.23 & 5.99 & 5.82 & 5.70 & 5.60 & 5.52 & 5.46 & 5.41 & 5.37 & 5.33 & 5.30 & 5.27 & 5.24 & 5.20 & 5.17 & 5.07 & 5.01 & 4.96 & 4.90 \\ 
  7 & 8.07 & 6.54 & 5.89 & 5.52 & 5.29 & 5.12 & 4.99 & 4.90 & 4.82 & 4.76 & 4.71 & 4.67 & 4.63 & 4.60 & 4.57 & 4.54 & 4.50 & 4.47 & 4.36 & 4.31 & 4.25 & 4.20 \\ 
  8 & 7.57 & 6.06 & 5.42 & 5.05 & 4.82 & 4.65 & 4.53 & 4.43 & 4.36 & 4.30 & 4.24 & 4.20 & 4.16 & 4.13 & 4.10 & 4.08 & 4.03 & 4.00 & 3.89 & 3.84 & 3.78 & 3.73 \\ 
  9 & 7.21 & 5.71 & 5.08 & 4.72 & 4.48 & 4.32 & 4.20 & 4.10 & 4.03 & 3.96 & 3.91 & 3.87 & 3.83 & 3.80 & 3.77 & 3.74 & 3.70 & 3.67 & 3.56 & 3.51 & 3.45 & 3.39 \\ 
  10 & 6.94 & 5.46 & 4.83 & 4.47 & 4.24 & 4.07 & 3.95 & 3.85 & 3.78 & 3.72 & 3.66 & 3.62 & 3.58 & 3.55 & 3.52 & 3.50 & 3.45 & 3.42 & 3.31 & 3.26 & 3.20 & 3.14 \\ 
  11 & 6.72 & 5.26 & 4.63 & 4.28 & 4.04 & 3.88 & 3.76 & 3.66 & 3.59 & 3.53 & 3.47 & 3.43 & 3.39 & 3.36 & 3.33 & 3.30 & 3.26 & 3.23 & 3.12 & 3.06 & 3.00 & 2.94 \\ 
  12 & 6.55 & 5.10 & 4.47 & 4.12 & 3.89 & 3.73 & 3.61 & 3.51 & 3.44 & 3.37 & 3.32 & 3.28 & 3.24 & 3.21 & 3.18 & 3.15 & 3.11 & 3.07 & 2.96 & 2.91 & 2.85 & 2.79 \\ 
  13 & 6.41 & 4.97 & 4.35 & 4.00 & 3.77 & 3.60 & 3.48 & 3.39 & 3.31 & 3.25 & 3.20 & 3.15 & 3.12 & 3.08 & 3.05 & 3.03 & 2.98 & 2.95 & 2.84 & 2.78 & 2.72 & 2.66 \\ 
  14 & 6.30 & 4.86 & 4.24 & 3.89 & 3.66 & 3.50 & 3.38 & 3.29 & 3.21 & 3.15 & 3.09 & 3.05 & 3.01 & 2.98 & 2.95 & 2.92 & 2.88 & 2.84 & 2.73 & 2.67 & 2.61 & 2.55 \\ 
  15 & 6.20 & 4.77 & 4.15 & 3.80 & 3.58 & 3.41 & 3.29 & 3.20 & 3.12 & 3.06 & 3.01 & 2.96 & 2.92 & 2.89 & 2.86 & 2.84 & 2.79 & 2.76 & 2.64 & 2.59 & 2.52 & 2.46 \\ 
  16 & 6.12 & 4.69 & 4.08 & 3.73 & 3.50 & 3.34 & 3.22 & 3.12 & 3.05 & 2.99 & 2.93 & 2.89 & 2.85 & 2.82 & 2.79 & 2.76 & 2.72 & 2.68 & 2.57 & 2.51 & 2.45 & 2.38 \\ 
  17 & 6.04 & 4.62 & 4.01 & 3.66 & 3.44 & 3.28 & 3.16 & 3.06 & 2.98 & 2.92 & 2.87 & 2.82 & 2.79 & 2.75 & 2.72 & 2.70 & 2.65 & 2.62 & 2.50 & 2.44 & 2.38 & 2.32 \\ 
  18 & 5.98 & 4.56 & 3.95 & 3.61 & 3.38 & 3.22 & 3.10 & 3.01 & 2.93 & 2.87 & 2.81 & 2.77 & 2.73 & 2.70 & 2.67 & 2.64 & 2.60 & 2.56 & 2.44 & 2.38 & 2.32 & 2.26 \\ 
  19 & 5.92 & 4.51 & 3.90 & 3.56 & 3.33 & 3.17 & 3.05 & 2.96 & 2.88 & 2.82 & 2.76 & 2.72 & 2.68 & 2.65 & 2.62 & 2.59 & 2.55 & 2.51 & 2.39 & 2.33 & 2.27 & 2.20 \\ 
  20 & 5.87 & 4.46 & 3.86 & 3.51 & 3.29 & 3.13 & 3.01 & 2.91 & 2.84 & 2.77 & 2.72 & 2.68 & 2.64 & 2.60 & 2.57 & 2.55 & 2.50 & 2.46 & 2.35 & 2.29 & 2.22 & 2.16 \\ 
  21 & 5.83 & 4.42 & 3.82 & 3.48 & 3.25 & 3.09 & 2.97 & 2.87 & 2.80 & 2.73 & 2.68 & 2.64 & 2.60 & 2.56 & 2.53 & 2.51 & 2.46 & 2.42 & 2.31 & 2.25 & 2.18 & 2.11 \\ 
  22 & 5.79 & 4.38 & 3.78 & 3.44 & 3.22 & 3.05 & 2.93 & 2.84 & 2.76 & 2.70 & 2.65 & 2.60 & 2.56 & 2.53 & 2.50 & 2.47 & 2.43 & 2.39 & 2.27 & 2.21 & 2.14 & 2.08 \\ 
  23 & 5.75 & 4.35 & 3.75 & 3.41 & 3.18 & 3.02 & 2.90 & 2.81 & 2.73 & 2.67 & 2.62 & 2.57 & 2.53 & 2.50 & 2.47 & 2.44 & 2.39 & 2.36 & 2.24 & 2.18 & 2.11 & 2.04 \\ 
  24 & 5.72 & 4.32 & 3.72 & 3.38 & 3.15 & 2.99 & 2.87 & 2.78 & 2.70 & 2.64 & 2.59 & 2.54 & 2.50 & 2.47 & 2.44 & 2.41 & 2.36 & 2.33 & 2.21 & 2.15 & 2.08 & 2.01 \\ 
  25 & 5.69 & 4.29 & 3.69 & 3.35 & 3.13 & 2.97 & 2.85 & 2.75 & 2.68 & 2.61 & 2.56 & 2.51 & 2.48 & 2.44 & 2.41 & 2.38 & 2.34 & 2.30 & 2.18 & 2.12 & 2.05 & 1.98 \\ 
  26 & 5.66 & 4.27 & 3.67 & 3.33 & 3.10 & 2.94 & 2.82 & 2.73 & 2.65 & 2.59 & 2.54 & 2.49 & 2.45 & 2.42 & 2.39 & 2.36 & 2.31 & 2.28 & 2.16 & 2.09 & 2.03 & 1.95 \\ 
  27 & 5.63 & 4.24 & 3.65 & 3.31 & 3.08 & 2.92 & 2.80 & 2.71 & 2.63 & 2.57 & 2.51 & 2.47 & 2.43 & 2.39 & 2.36 & 2.34 & 2.29 & 2.25 & 2.13 & 2.07 & 2.00 & 1.93 \\ 
  28 & 5.61 & 4.22 & 3.63 & 3.29 & 3.06 & 2.90 & 2.78 & 2.69 & 2.61 & 2.55 & 2.49 & 2.45 & 2.41 & 2.37 & 2.34 & 2.32 & 2.27 & 2.23 & 2.11 & 2.05 & 1.98 & 1.91 \\ 
  29 & 5.59 & 4.20 & 3.61 & 3.27 & 3.04 & 2.88 & 2.76 & 2.67 & 2.59 & 2.53 & 2.48 & 2.43 & 2.39 & 2.36 & 2.32 & 2.30 & 2.25 & 2.21 & 2.09 & 2.03 & 1.96 & 1.89 \\ 
  30 & 5.57 & 4.18 & 3.59 & 3.25 & 3.03 & 2.87 & 2.75 & 2.65 & 2.57 & 2.51 & 2.46 & 2.41 & 2.37 & 2.34 & 2.31 & 2.28 & 2.23 & 2.20 & 2.07 & 2.01 & 1.94 & 1.87 \\ 
  40 & 5.42 & 4.05 & 3.46 & 3.13 & 2.90 & 2.74 & 2.62 & 2.53 & 2.45 & 2.39 & 2.33 & 2.29 & 2.25 & 2.21 & 2.18 & 2.15 & 2.11 & 2.07 & 1.94 & 1.88 & 1.80 & 1.72 \\ 
  60 & 5.29 & 3.93 & 3.34 & 3.01 & 2.79 & 2.63 & 2.51 & 2.41 & 2.33 & 2.27 & 2.22 & 2.17 & 2.13 & 2.09 & 2.06 & 2.03 & 1.98 & 1.94 & 1.82 & 1.74 & 1.67 & 1.58 \\ 
  120 & 5.15 & 3.80 & 3.23 & 2.89 & 2.67 & 2.52 & 2.39 & 2.30 & 2.22 & 2.16 & 2.10 & 2.05 & 2.01 & 1.98 & 1.94 & 1.92 & 1.87 & 1.82 & 1.69 & 1.61 & 1.53 & 1.43 \\ 
   \hline
\end{tabular}
\end{table}

\end{landscape}


\newpage


\begin{landscape}

\begin{minipage}[l]{\textwidth}
\begin{Huge}
Distribuição F de Snedecor ($1\%$)
\end{Huge}
\par
\vspace{2em}
A tabela fornece o valor de $q$, tal que $\mathbb{P}(F_{(\nu_1, \nu_2)} \geq q) = 0.01$, em que $\nu_1$ são os graus de liberdade do numerador e $\nu_2$ os graus de liberdade do denominador. 
\par\bigskip
\href{https://creativecommons.org/licenses/by-sa/4.0/deed.pt_BR}{\includegraphics[height=1em]{cc-by-sa.pdf}}
\href{https://rea-mecc.github.io}{REA-MECC}
\end{minipage}
\hfill
\begin{minipage}[c]{.35\textwidth}
\includegraphics[height=.9\linewidth]{plotf001.pdf}
\end{minipage}

\addtolength{\tabcolsep}{-0.4pt}
% Tabela gerada pelo R com pequenas modificações feitas manualmente
\begin{table}[H]
\centering
\scriptsize
\begin{tabular}{rcccccccccccccccccccccc}
  \hline
 & \multicolumn{22}{c}{$\nu_1$} \\

$\nu_2$ & 1 & 2 & 3 & 4 & 5 & 6 & 7 & 8 & 9 & 10 & 11 & 12 & 13 & 14 & 15 & 16 & 18 & 20 & 30 & 40 & 60 & 120 \\ 
  \hline
2 & 98.50 & 99.00 & 99.17 & 99.25 & 99.30 & 99.33 & 99.36 & 99.37 & 99.39 & 99.40 & 99.41 & 99.42 & 99.42 & 99.43 & 99.43 & 99.44 & 99.44 & 99.45 & 99.47 & 99.47 & 99.48 & 99.49 \\ 
  3 & 34.12 & 30.82 & 29.46 & 28.71 & 28.24 & 27.91 & 27.67 & 27.49 & 27.35 & 27.23 & 27.13 & 27.05 & 26.98 & 26.92 & 26.87 & 26.83 & 26.75 & 26.69 & 26.50 & 26.41 & 26.32 & 26.22 \\ 
  4 & 21.20 & 18.00 & 16.69 & 15.98 & 15.52 & 15.21 & 14.98 & 14.80 & 14.66 & 14.55 & 14.45 & 14.37 & 14.31 & 14.25 & 14.20 & 14.15 & 14.08 & 14.02 & 13.84 & 13.75 & 13.65 & 13.56 \\ 
  5 & 16.26 & 13.27 & 12.06 & 11.39 & 10.97 & 10.67 & 10.46 & 10.29 & 10.16 & 10.05 & 9.96 & 9.89 & 9.82 & 9.77 & 9.72 & 9.68 & 9.61 & 9.55 & 9.38 & 9.29 & 9.20 & 9.11 \\ 
  6 & 13.75 & 10.92 & 9.78 & 9.15 & 8.75 & 8.47 & 8.26 & 8.10 & 7.98 & 7.87 & 7.79 & 7.72 & 7.66 & 7.60 & 7.56 & 7.52 & 7.45 & 7.40 & 7.23 & 7.14 & 7.06 & 6.97 \\ 
  7 & 12.25 & 9.55 & 8.45 & 7.85 & 7.46 & 7.19 & 6.99 & 6.84 & 6.72 & 6.62 & 6.54 & 6.47 & 6.41 & 6.36 & 6.31 & 6.28 & 6.21 & 6.16 & 5.99 & 5.91 & 5.82 & 5.74 \\ 
  8 & 11.26 & 8.65 & 7.59 & 7.01 & 6.63 & 6.37 & 6.18 & 6.03 & 5.91 & 5.81 & 5.73 & 5.67 & 5.61 & 5.56 & 5.52 & 5.48 & 5.41 & 5.36 & 5.20 & 5.12 & 5.03 & 4.95 \\ 
  9 & 10.56 & 8.02 & 6.99 & 6.42 & 6.06 & 5.80 & 5.61 & 5.47 & 5.35 & 5.26 & 5.18 & 5.11 & 5.05 & 5.01 & 4.96 & 4.92 & 4.86 & 4.81 & 4.65 & 4.57 & 4.48 & 4.40 \\ 
  10 & 10.04 & 7.56 & 6.55 & 5.99 & 5.64 & 5.39 & 5.20 & 5.06 & 4.94 & 4.85 & 4.77 & 4.71 & 4.65 & 4.60 & 4.56 & 4.52 & 4.46 & 4.41 & 4.25 & 4.17 & 4.08 & 4.00 \\ 
  11 & 9.65 & 7.21 & 6.22 & 5.67 & 5.32 & 5.07 & 4.89 & 4.74 & 4.63 & 4.54 & 4.46 & 4.40 & 4.34 & 4.29 & 4.25 & 4.21 & 4.15 & 4.10 & 3.94 & 3.86 & 3.78 & 3.69 \\ 
  12 & 9.33 & 6.93 & 5.95 & 5.41 & 5.06 & 4.82 & 4.64 & 4.50 & 4.39 & 4.30 & 4.22 & 4.16 & 4.10 & 4.05 & 4.01 & 3.97 & 3.91 & 3.86 & 3.70 & 3.62 & 3.54 & 3.45 \\ 
  13 & 9.07 & 6.70 & 5.74 & 5.21 & 4.86 & 4.62 & 4.44 & 4.30 & 4.19 & 4.10 & 4.02 & 3.96 & 3.91 & 3.86 & 3.82 & 3.78 & 3.72 & 3.66 & 3.51 & 3.43 & 3.34 & 3.25 \\ 
  14 & 8.86 & 6.51 & 5.56 & 5.04 & 4.69 & 4.46 & 4.28 & 4.14 & 4.03 & 3.94 & 3.86 & 3.80 & 3.75 & 3.70 & 3.66 & 3.62 & 3.56 & 3.51 & 3.35 & 3.27 & 3.18 & 3.09 \\ 
  15 & 8.68 & 6.36 & 5.42 & 4.89 & 4.56 & 4.32 & 4.14 & 4.00 & 3.89 & 3.80 & 3.73 & 3.67 & 3.61 & 3.56 & 3.52 & 3.49 & 3.42 & 3.37 & 3.21 & 3.13 & 3.05 & 2.96 \\ 
  16 & 8.53 & 6.23 & 5.29 & 4.77 & 4.44 & 4.20 & 4.03 & 3.89 & 3.78 & 3.69 & 3.62 & 3.55 & 3.50 & 3.45 & 3.41 & 3.37 & 3.31 & 3.26 & 3.10 & 3.02 & 2.93 & 2.84 \\ 
  17 & 8.40 & 6.11 & 5.18 & 4.67 & 4.34 & 4.10 & 3.93 & 3.79 & 3.68 & 3.59 & 3.52 & 3.46 & 3.40 & 3.35 & 3.31 & 3.27 & 3.21 & 3.16 & 3.00 & 2.92 & 2.83 & 2.75 \\ 
  18 & 8.29 & 6.01 & 5.09 & 4.58 & 4.25 & 4.01 & 3.84 & 3.71 & 3.60 & 3.51 & 3.43 & 3.37 & 3.32 & 3.27 & 3.23 & 3.19 & 3.13 & 3.08 & 2.92 & 2.84 & 2.75 & 2.66 \\ 
  19 & 8.18 & 5.93 & 5.01 & 4.50 & 4.17 & 3.94 & 3.77 & 3.63 & 3.52 & 3.43 & 3.36 & 3.30 & 3.24 & 3.19 & 3.15 & 3.12 & 3.05 & 3.00 & 2.84 & 2.76 & 2.67 & 2.58 \\ 
  20 & 8.10 & 5.85 & 4.94 & 4.43 & 4.10 & 3.87 & 3.70 & 3.56 & 3.46 & 3.37 & 3.29 & 3.23 & 3.18 & 3.13 & 3.09 & 3.05 & 2.99 & 2.94 & 2.78 & 2.69 & 2.61 & 2.52 \\ 
  21 & 8.02 & 5.78 & 4.87 & 4.37 & 4.04 & 3.81 & 3.64 & 3.51 & 3.40 & 3.31 & 3.24 & 3.17 & 3.12 & 3.07 & 3.03 & 2.99 & 2.93 & 2.88 & 2.72 & 2.64 & 2.55 & 2.46 \\ 
  22 & 7.95 & 5.72 & 4.82 & 4.31 & 3.99 & 3.76 & 3.59 & 3.45 & 3.35 & 3.26 & 3.18 & 3.12 & 3.07 & 3.02 & 2.98 & 2.94 & 2.88 & 2.83 & 2.67 & 2.58 & 2.50 & 2.40 \\ 
  23 & 7.88 & 5.66 & 4.76 & 4.26 & 3.94 & 3.71 & 3.54 & 3.41 & 3.30 & 3.21 & 3.14 & 3.07 & 3.02 & 2.97 & 2.93 & 2.89 & 2.83 & 2.78 & 2.62 & 2.54 & 2.45 & 2.35 \\ 
  24 & 7.82 & 5.61 & 4.72 & 4.22 & 3.90 & 3.67 & 3.50 & 3.36 & 3.26 & 3.17 & 3.09 & 3.03 & 2.98 & 2.93 & 2.89 & 2.85 & 2.79 & 2.74 & 2.58 & 2.49 & 2.40 & 2.31 \\ 
  25 & 7.77 & 5.57 & 4.68 & 4.18 & 3.85 & 3.63 & 3.46 & 3.32 & 3.22 & 3.13 & 3.06 & 2.99 & 2.94 & 2.89 & 2.85 & 2.81 & 2.75 & 2.70 & 2.54 & 2.45 & 2.36 & 2.27 \\ 
  26 & 7.72 & 5.53 & 4.64 & 4.14 & 3.82 & 3.59 & 3.42 & 3.29 & 3.18 & 3.09 & 3.02 & 2.96 & 2.90 & 2.86 & 2.81 & 2.78 & 2.72 & 2.66 & 2.50 & 2.42 & 2.33 & 2.23 \\ 
  27 & 7.68 & 5.49 & 4.60 & 4.11 & 3.78 & 3.56 & 3.39 & 3.26 & 3.15 & 3.06 & 2.99 & 2.93 & 2.87 & 2.82 & 2.78 & 2.75 & 2.68 & 2.63 & 2.47 & 2.38 & 2.29 & 2.20 \\ 
  28 & 7.64 & 5.45 & 4.57 & 4.07 & 3.75 & 3.53 & 3.36 & 3.23 & 3.12 & 3.03 & 2.96 & 2.90 & 2.84 & 2.79 & 2.75 & 2.72 & 2.65 & 2.60 & 2.44 & 2.35 & 2.26 & 2.17 \\ 
  29 & 7.60 & 5.42 & 4.54 & 4.04 & 3.73 & 3.50 & 3.33 & 3.20 & 3.09 & 3.00 & 2.93 & 2.87 & 2.81 & 2.77 & 2.73 & 2.69 & 2.63 & 2.57 & 2.41 & 2.33 & 2.23 & 2.14 \\ 
  30 & 7.56 & 5.39 & 4.51 & 4.02 & 3.70 & 3.47 & 3.30 & 3.17 & 3.07 & 2.98 & 2.91 & 2.84 & 2.79 & 2.74 & 2.70 & 2.66 & 2.60 & 2.55 & 2.39 & 2.30 & 2.21 & 2.11 \\ 
  40 & 7.31 & 5.18 & 4.31 & 3.83 & 3.51 & 3.29 & 3.12 & 2.99 & 2.89 & 2.80 & 2.73 & 2.66 & 2.61 & 2.56 & 2.52 & 2.48 & 2.42 & 2.37 & 2.20 & 2.11 & 2.02 & 1.92 \\ 
  60 & 7.08 & 4.98 & 4.13 & 3.65 & 3.34 & 3.12 & 2.95 & 2.82 & 2.72 & 2.63 & 2.56 & 2.50 & 2.44 & 2.39 & 2.35 & 2.31 & 2.25 & 2.20 & 2.03 & 1.94 & 1.84 & 1.73 \\ 
  120 & 6.85 & 4.79 & 3.95 & 3.48 & 3.17 & 2.96 & 2.79 & 2.66 & 2.56 & 2.47 & 2.40 & 2.34 & 2.28 & 2.23 & 2.19 & 2.15 & 2.09 & 2.03 & 1.86 & 1.76 & 1.66 & 1.53 \\ 
   \hline
\end{tabular}
\end{table}

\end{landscape}


\newpage


\begin{landscape}

\begin{minipage}[l]{\textwidth}
\begin{Huge}
Distribuição F de Snedecor ($0.5\%$)
\end{Huge}
\par
\vspace{2em}
A tabela fornece o valor de $q$, tal que $\mathbb{P}(F_{(\nu_1, \nu_2)} \geq q) = 0.005$, em que $\nu_1$ são os graus de liberdade do numerador e $\nu_2$ os graus de liberdade do denominador. 
\par\bigskip
\href{https://creativecommons.org/licenses/by-sa/4.0/deed.pt_BR}{\includegraphics[height=1em]{cc-by-sa.pdf}}
\href{https://rea-mecc.github.io}{REA-MECC}
\end{minipage}
\hfill
\begin{minipage}[c]{.35\textwidth}
\includegraphics[height=.9\linewidth]{plotf0005.pdf}
\end{minipage}

\addtolength{\tabcolsep}{-1.9pt}
% Tabela gerada pelo R com pequenas modificações feitas manualmente
\begin{table}[H]
\centering
\scriptsize
\begin{tabular}{rcccccccccccccccccccccc}
  \hline
 & \multicolumn{22}{c}{$\nu_1$} \\

$\nu_2$ & 1 & 2 & 3 & 4 & 5 & 6 & 7 & 8 & 9 & 10 & 11 & 12 & 13 & 14 & 15 & 16 & 18 & 20 & 30 & 40 & 60 & 120 \\ 
  \hline
2 & 198.50 & 199.00 & 199.17 & 199.25 & 199.30 & 199.33 & 199.36 & 199.37 & 199.39 & 199.40 & 199.41 & 199.42 & 199.42 & 199.43 & 199.43 & 199.44 & 199.44 & 199.45 & 199.47 & 199.47 & 199.48 & 199.49 \\ 
  3 & 55.55 & 49.80 & 47.47 & 46.19 & 45.39 & 44.84 & 44.43 & 44.13 & 43.88 & 43.69 & 43.52 & 43.39 & 43.27 & 43.17 & 43.08 & 43.01 & 42.88 & 42.78 & 42.47 & 42.31 & 42.15 & 41.99 \\ 
  4 & 31.33 & 26.28 & 24.26 & 23.15 & 22.46 & 21.97 & 21.62 & 21.35 & 21.14 & 20.97 & 20.82 & 20.70 & 20.60 & 20.51 & 20.44 & 20.37 & 20.26 & 20.17 & 19.89 & 19.75 & 19.61 & 19.47 \\ 
  5 & 22.78 & 18.31 & 16.53 & 15.56 & 14.94 & 14.51 & 14.20 & 13.96 & 13.77 & 13.62 & 13.49 & 13.38 & 13.29 & 13.21 & 13.15 & 13.09 & 12.98 & 12.90 & 12.66 & 12.53 & 12.40 & 12.27 \\ 
  6 & 18.63 & 14.54 & 12.92 & 12.03 & 11.46 & 11.07 & 10.79 & 10.57 & 10.39 & 10.25 & 10.13 & 10.03 & 9.95 & 9.88 & 9.81 & 9.76 & 9.66 & 9.59 & 9.36 & 9.24 & 9.12 & 9.00 \\ 
  7 & 16.24 & 12.40 & 10.88 & 10.05 & 9.52 & 9.16 & 8.89 & 8.68 & 8.51 & 8.38 & 8.27 & 8.18 & 8.10 & 8.03 & 7.97 & 7.91 & 7.83 & 7.75 & 7.53 & 7.42 & 7.31 & 7.19 \\ 
  8 & 14.69 & 11.04 & 9.60 & 8.81 & 8.30 & 7.95 & 7.69 & 7.50 & 7.34 & 7.21 & 7.10 & 7.01 & 6.94 & 6.87 & 6.81 & 6.76 & 6.68 & 6.61 & 6.40 & 6.29 & 6.18 & 6.06 \\ 
  9 & 13.61 & 10.11 & 8.72 & 7.96 & 7.47 & 7.13 & 6.88 & 6.69 & 6.54 & 6.42 & 6.31 & 6.23 & 6.15 & 6.09 & 6.03 & 5.98 & 5.90 & 5.83 & 5.62 & 5.52 & 5.41 & 5.30 \\ 
  10 & 12.83 & 9.43 & 8.08 & 7.34 & 6.87 & 6.54 & 6.30 & 6.12 & 5.97 & 5.85 & 5.75 & 5.66 & 5.59 & 5.53 & 5.47 & 5.42 & 5.34 & 5.27 & 5.07 & 4.97 & 4.86 & 4.75 \\ 
  11 & 12.23 & 8.91 & 7.60 & 6.88 & 6.42 & 6.10 & 5.86 & 5.68 & 5.54 & 5.42 & 5.32 & 5.24 & 5.16 & 5.10 & 5.05 & 5.00 & 4.92 & 4.86 & 4.65 & 4.55 & 4.45 & 4.34 \\ 
  12 & 11.75 & 8.51 & 7.23 & 6.52 & 6.07 & 5.76 & 5.52 & 5.35 & 5.20 & 5.09 & 4.99 & 4.91 & 4.84 & 4.77 & 4.72 & 4.67 & 4.59 & 4.53 & 4.33 & 4.23 & 4.12 & 4.01 \\ 
  13 & 11.37 & 8.19 & 6.93 & 6.23 & 5.79 & 5.48 & 5.25 & 5.08 & 4.94 & 4.82 & 4.72 & 4.64 & 4.57 & 4.51 & 4.46 & 4.41 & 4.33 & 4.27 & 4.07 & 3.97 & 3.87 & 3.76 \\ 
  14 & 11.06 & 7.92 & 6.68 & 6.00 & 5.56 & 5.26 & 5.03 & 4.86 & 4.72 & 4.60 & 4.51 & 4.43 & 4.36 & 4.30 & 4.25 & 4.20 & 4.12 & 4.06 & 3.86 & 3.76 & 3.66 & 3.55 \\ 
  15 & 10.80 & 7.70 & 6.48 & 5.80 & 5.37 & 5.07 & 4.85 & 4.67 & 4.54 & 4.42 & 4.33 & 4.25 & 4.18 & 4.12 & 4.07 & 4.02 & 3.95 & 3.88 & 3.69 & 3.58 & 3.48 & 3.37 \\ 
  16 & 10.58 & 7.51 & 6.30 & 5.64 & 5.21 & 4.91 & 4.69 & 4.52 & 4.38 & 4.27 & 4.18 & 4.10 & 4.03 & 3.97 & 3.92 & 3.87 & 3.80 & 3.73 & 3.54 & 3.44 & 3.33 & 3.22 \\ 
  17 & 10.38 & 7.35 & 6.16 & 5.50 & 5.07 & 4.78 & 4.56 & 4.39 & 4.25 & 4.14 & 4.05 & 3.97 & 3.90 & 3.84 & 3.79 & 3.75 & 3.67 & 3.61 & 3.41 & 3.31 & 3.21 & 3.10 \\ 
  18 & 10.22 & 7.21 & 6.03 & 5.37 & 4.96 & 4.66 & 4.44 & 4.28 & 4.14 & 4.03 & 3.94 & 3.86 & 3.79 & 3.73 & 3.68 & 3.64 & 3.56 & 3.50 & 3.30 & 3.20 & 3.10 & 2.99 \\ 
  19 & 10.07 & 7.09 & 5.92 & 5.27 & 4.85 & 4.56 & 4.34 & 4.18 & 4.04 & 3.93 & 3.84 & 3.76 & 3.70 & 3.64 & 3.59 & 3.54 & 3.46 & 3.40 & 3.21 & 3.11 & 3.00 & 2.89 \\ 
  20 & 9.94 & 6.99 & 5.82 & 5.17 & 4.76 & 4.47 & 4.26 & 4.09 & 3.96 & 3.85 & 3.76 & 3.68 & 3.61 & 3.55 & 3.50 & 3.46 & 3.38 & 3.32 & 3.12 & 3.02 & 2.92 & 2.81 \\ 
  21 & 9.83 & 6.89 & 5.73 & 5.09 & 4.68 & 4.39 & 4.18 & 4.01 & 3.88 & 3.77 & 3.68 & 3.60 & 3.54 & 3.48 & 3.43 & 3.38 & 3.31 & 3.24 & 3.05 & 2.95 & 2.84 & 2.73 \\ 
  22 & 9.73 & 6.81 & 5.65 & 5.02 & 4.61 & 4.32 & 4.11 & 3.94 & 3.81 & 3.70 & 3.61 & 3.54 & 3.47 & 3.41 & 3.36 & 3.31 & 3.24 & 3.18 & 2.98 & 2.88 & 2.77 & 2.66 \\ 
  23 & 9.63 & 6.73 & 5.58 & 4.95 & 4.54 & 4.26 & 4.05 & 3.88 & 3.75 & 3.64 & 3.55 & 3.47 & 3.41 & 3.35 & 3.30 & 3.25 & 3.18 & 3.12 & 2.92 & 2.82 & 2.71 & 2.60 \\ 
  24 & 9.55 & 6.66 & 5.52 & 4.89 & 4.49 & 4.20 & 3.99 & 3.83 & 3.69 & 3.59 & 3.50 & 3.42 & 3.35 & 3.30 & 3.25 & 3.20 & 3.12 & 3.06 & 2.87 & 2.77 & 2.66 & 2.55 \\ 
  25 & 9.48 & 6.60 & 5.46 & 4.84 & 4.43 & 4.15 & 3.94 & 3.78 & 3.64 & 3.54 & 3.45 & 3.37 & 3.30 & 3.25 & 3.20 & 3.15 & 3.08 & 3.01 & 2.82 & 2.72 & 2.61 & 2.50 \\ 
  26 & 9.41 & 6.54 & 5.41 & 4.79 & 4.38 & 4.10 & 3.89 & 3.73 & 3.60 & 3.49 & 3.40 & 3.33 & 3.26 & 3.20 & 3.15 & 3.11 & 3.03 & 2.97 & 2.77 & 2.67 & 2.56 & 2.45 \\ 
  27 & 9.34 & 6.49 & 5.36 & 4.74 & 4.34 & 4.06 & 3.85 & 3.69 & 3.56 & 3.45 & 3.36 & 3.28 & 3.22 & 3.16 & 3.11 & 3.07 & 2.99 & 2.93 & 2.73 & 2.63 & 2.52 & 2.41 \\ 
  28 & 9.28 & 6.44 & 5.32 & 4.70 & 4.30 & 4.02 & 3.81 & 3.65 & 3.52 & 3.41 & 3.32 & 3.25 & 3.18 & 3.12 & 3.07 & 3.03 & 2.95 & 2.89 & 2.69 & 2.59 & 2.48 & 2.37 \\ 
  29 & 9.23 & 6.40 & 5.28 & 4.66 & 4.26 & 3.98 & 3.77 & 3.61 & 3.48 & 3.38 & 3.29 & 3.21 & 3.15 & 3.09 & 3.04 & 2.99 & 2.92 & 2.86 & 2.66 & 2.56 & 2.45 & 2.33 \\ 
  30 & 9.18 & 6.35 & 5.24 & 4.62 & 4.23 & 3.95 & 3.74 & 3.58 & 3.45 & 3.34 & 3.25 & 3.18 & 3.11 & 3.06 & 3.01 & 2.96 & 2.89 & 2.82 & 2.63 & 2.52 & 2.42 & 2.30 \\ 
  40 & 8.83 & 6.07 & 4.98 & 4.37 & 3.99 & 3.71 & 3.51 & 3.35 & 3.22 & 3.12 & 3.03 & 2.95 & 2.89 & 2.83 & 2.78 & 2.74 & 2.66 & 2.60 & 2.40 & 2.30 & 2.18 & 2.06 \\ 
  60 & 8.49 & 5.79 & 4.73 & 4.14 & 3.76 & 3.49 & 3.29 & 3.13 & 3.01 & 2.90 & 2.82 & 2.74 & 2.68 & 2.62 & 2.57 & 2.53 & 2.45 & 2.39 & 2.19 & 2.08 & 1.96 & 1.83 \\ 
  120 & 8.18 & 5.54 & 4.50 & 3.92 & 3.55 & 3.28 & 3.09 & 2.93 & 2.81 & 2.71 & 2.62 & 2.54 & 2.48 & 2.42 & 2.37 & 2.33 & 2.25 & 2.19 & 1.98 & 1.87 & 1.75 & 1.61 \\ 
   \hline
\end{tabular}
\end{table}

\end{landscape}



\end{document}

